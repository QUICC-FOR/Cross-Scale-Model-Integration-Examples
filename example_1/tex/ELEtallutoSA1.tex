\documentclass[11pt]{article}

%% ----------------------------------
%
%     Packages to be used in the final manuscript
%
%% ----------------------------------


\usepackage[margin=1in]{geometry}

% use proper unicode fonts
\usepackage[T1]{fontenc}
\usepackage[utf8]{inputenc}

\usepackage{xcolor}
\definecolor{light-gray}{gray}{0.85}
\fboxsep2pt
\newcommand{\code}[1]{\colorbox{light-gray}{\ttfamily #1}}

%% ----------------------------------
%
%     END PREAMBLE
%
%% ----------------------------------

\begin{document}
%% ----------------------------------
%
%     TITLE PAGE
%
%% ----------------------------------


%% ----------------------------------
%
%     INTRODUCTION
%
%% ----------------------------------

{\Large \flushleft
Appendix S1: Code, data, and information for model integration example 1
}

\section{Building and running the model}
Building the model requires an installation of R, JAGS, and rjags (see \textbf{Dependencies} below for information on all software required to fully build and run the model).
For users familiar with Gnu Make, a makefile is provided as a guide.
To produce this PDF, use \code{make SI}. 
To run the analyses, use \code{make analysis}.
To produce the figures, use \code{make figures}.
Additionally, global settings for controlling the analysis and the production of figures are contained in the file \code{ex1\_globals.r}.
There are a total of 6 steps to complete the analysis and build all figures.
Complete details are documented in the included source files, and summaries are provided below.

\subsection{Run the naive model (no integration)}
\code{Rscript 1-m1.r} or \\
\code{make dat/ex1\_m1.rdata}

This script will initiate a JAGS model that estimates the parameters of the metamodel with no conditioning on the information from the sub-model (i.e., the naive model).
We first generated a simulated dataset of 100 presence-absence points at randomized locations in ecological space (i.e., temperature and precipitation; see main text) using the \code{spsample} function from package \code{sp} in R.
This function generates points with a specified amount of spatial clustering (ranging from 0 to 1, where 1 represents complete spatial independence); we selected a value of 0.2 for our analysis.
Presence or absence at each point was determined by randomly sampling from a Bernoulli distribution, where the probability was selected using a pre-determined function of temperature and precipitation.
We then fit the naive model in JAGS as a logistic regression, with both linear and quadratic terms for both temperature and precipitation and using uninformative priors (Normal with \(\mu = 0\) and \(\sigma = 10000\)).
We discarded the first 5000 MCMC samples as burnin, and then collected an additional 2000 samples for analysis.
The final sample size was selected to provide for relatively rapid computational time; the addition of longer final samples or extended burnin periods had no effect on the results.

\subsection{Run the sub-model}
\code{Rscript 2-m2.r} or \\
\code{make dat/ex1\_m2.rdata}

This script computes the predicted probability of presence as a function of precipitation, using the experimental data and the theoretical prediction that the species will be present when the population growth rate is greater than 0.
For each of the five experimental treatments, we computed the probability of presence as the integral of the normal density from 0 to infinity, where the mean of the normal distribution was equal to the average population growth rate at each precipitation treatment and the standard deviation was the corresponding standard error for each treatment.
We then fit these data to the same model, priors, and sample sizes as those from the previous step.

\subsection{Run the metamodel}
\code{Rscript 3-mm.r} or \\
\code{make dat/ex1\_mm.rdata}

Fitting the metamodel is simply a matter of repeating the exact procedure for fitting the naive model, but using informative priors generated from the sub-model step.
Because the sub-model considered only precipitation, and because the response was highly correlated to precipitation, we expected the submodel to produce highly precise estimates of the parameters for the effect of precipitation on the probability of presence.
However, the sub-model was quite simplistic, and thus likely over-estimates its precision when considering the species range wholistically (as is done with the metamodel).
Thus, we applied a prior weight of 0.05 to the sub-model.
This allowed the sub-model to inform the integrated model without dominating the results.

\subsection{Produce figures}
\code{Rscript 4-makeSamplingFig.r ex1\_Sampling.pdf} \\
\code{Rscript 5-makePrecipFig.r ex1\_precip.pdf} \\
\code{Rscript 6-makeMapFig.r ex1\_map.pdf}; or \\
\code{make figures}


\section{Dependencies}
All scripts were prepared in R and tested under R version 3.0.3.
The MCMC analyses were performed using the JAGS (Just Another Gibbs Sampler) software package, version 3.4.0.
Additionally, the \code{rjags}, \code{coda}, and \code{sp} packages are required to complete the analysis.
\end{document}